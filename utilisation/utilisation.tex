\chapter{Utilisation de GEOM\_LUKS}

\section{Compatibilité entre Linux et FreeBSD}

\subsection{Le système de fichiers}
Nous avons donc implémenté une première version d'un module noyau et de son
utilitaire pour FreeBSD qui permet la lecture et l'écriture sur un volume chiffré
avec le format LUKS1.0. La version 2.0 de LUKS est encore en développement,
pour arriver à sa version finale, bien que déjà disponible sous Linux. Pour
pouvoir utiliser la partition au format LUKS sous FreeBSD il faut un système de
fichier qui soit supporté par les deux systèmes d'exploitation. En listant
les systèmes de fichiers disponibles, on peut noter que {\em XFS, EXT2, openZFS,
FAT32, et NTFS} sont disponibles sous Linux et FreeBSD.

\subsection{Le démarrage}
Nous n'avons pas implémenter la partie s'occupant du démarrage sur une partition
chiffrée avec {\em LUKS}. Cependant pour démarrer sur un système dans lequel le
{\em /boot} est séparé, il suffirait de peu de modification au code actuel pour
permettre au noyau lors de son chargement d'ouvrir le {\em /} chiffré avec LUKS.


\subsection{Partage de fichiers et haute disponibilité}
On peut donc imaginer des cas d'utilisation de LUKS sous FreeBSD permettant par
exemple de partager des fichiers entre Linux et FreeBSD, et donc de faciliter
l'installation de dual-boot Linux-FreeBSD. On peut également imaginer disposer
de systèmes Linux et FreeBSD pour réaliser la même opération sur un stockage
réseau qui serait chiffré, avec LUKS, permettant d'accroître la disponiblité du
système en utilisant deux systèmes d'exploitation différents tout en conservant
la propriété de confidentialité, en permettant uniquement à la machine qui
utilise le disque de pouvoir le lire (et donc interdisant la machine hébergeant
le disque de pouvoir lire son contenu).
