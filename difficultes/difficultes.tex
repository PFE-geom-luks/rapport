\chapter{Difficultés rencontrées}

\section{Développement noyau et débogage}

\subsection{Développement noyau}

\paragraph{}Nous avons du implémenter la majorité des fonctions dans le code du
module noyau que nous avons créé à partir de {\em GELI}. Il a donc fallu se
familiariser avec les différences entre noyau et espace utilisateur, on peut
par exemple citer l'allocation dynamique ou encore les fonctions de chiffrement
et de déchiffrement qui diffèrent.

\paragraph{}
Dans l'espace utilisateur la bibliothèque {\em Openssl} fournit la plupart des
fonctions de chiffrement et de hashage, tandis que dans le noyau nous avons du
utiliser l'API {\em Opencrypto}. De même, nous avons du nous familiariser avec
l'API que propose {\em GEOM} pour comprendre les différents champs des
structures et fonctions utilisées. Concernant les algorithmes de hashage, l'API
{\em Opencrypto} fourni les fonctions et types pour chaque algorithme, et c'est
donc au développeur de s'adapter selon l'algorithme à utiliser. Du fait que
les types pour chaque algorithme de hashage diffèrent (SHA1, SHA256, SHA512),
on est contraint de répéter du code pour chaque algorithme.

\subsection{Débogage}
\paragraph{}
L'une des plus grosse difficultés provenait du fait, qu'en mode noyau,
peu d'erreurs sont permises. Par exemple nous avons souvent du faire face à
des erreurs de page ou des débordements mémoire qui ne sont alors pas
rattrapables par le noyau, qui causait alors un plantage du noyau. Nous avons
tout de même pris soin d'activer le débogage noyau dans FreeBSD, ce qui nous
permettaient de récupérer une image mémoire du noyau que l'on pouvait exploiter
ultérieurement avec un débogueur noyau.

\paragraph{}Dans les cas où il n'y avait pas de plantage, l'utilisation de
{\em printf} nous permettait de déboguer le programme en affichant la valeur
de certaines variables, qui apparaissaient alors dans les messages envoyées par
le noyau, que l'on peut consulter avec la commande {\em dmesg} sous la majorité
des systèmes UNIX dont FreeBSD.




