\chapter{Conclusion}
\section{Module noyau et utilitaire GEOM\_LUKS}

\paragraph{}
Pour l'instant le module noyau et l'utilitaire développés supportent l'ouverture
d'un disque chiffré avec {\em LUKS}, et donc l'écriture et la lecture d'un
volume qui est chiffré avec le standard. Cependant le module ne supporte
actuellement pas le standard dans son intégralité.

%TODO: Ecrire un paragraphe ou tableau ou liste de ce qu'il manque ou ce qu'on a en fonction de ce qui est le plus court

\section{Suite du projet}
\paragraph{}
Pour arriver à pouvoir intégrer le module noyau et l'utilitaire dans la branche
principale de FreeBSD, il reste cependant quelques étapes essentielles. Il
faudrait probablement dans un premier temps factoriser le code, enlever les
fonctions de {\em GELI} qui ne sont plus utiles, simplifier les fonctions en
enlevant le code qui était utile sous {\em GELI} mais ne l'est plus sous
{\em GLUKS}, on peut citer notamment tout ce qui est en rapport avec
l'authentification des données sur le disque, ou encore les fonctions s'occupant
de la dérivation de la clé de chiffrement. Par la suite, on pourrait modifier,
voire se passer des structures \texttt{g\_luks\_metadata} et
\texttt{g\_luks\_softc} héritées de {\em GELI} pour n'utiliser que celles de
{\em GLUKS}. Il faudrait également développer les scripts de tests, probablement
en adaptant ceux de {\em   GELI}, car l'utilitaire {\em geom\_luks} est très
proche de {\em geom\_eli}. Enfin il faudrait s'assurer du respect des
conventions de style de FreeBSD pour le noyau, et faire réaliser une revue de
code par les développeurs de FreeBSD.

\paragraph{}
Concernant l'utilitaire, il permet actuellement d'attacher (montrer le disque 
déchiffré au système pour permettre la lecture et l'écriture) et détacher un
volume chiffré. Il permet
également d'afficher les métadonnées stockées dans l'en-tête {\em LUKS}, ainsi que de
lister les volumes actuellement attaché. Il reste donc les opérations de
création, changement de clé, destruction. Il manque également le support de 
fichiers clé, qui permettent de remplacer ou compléter la phrase de passe.

\paragraph{}
Enfin pour son intégration dans FreeBSD, il serait possible d'ajouter le support
du démarrage sur un volume chiffré avec {\em LUKS}, en ajoutant le code nécessaire
au module noyau ainsi qu'au chargeur et éventuellement au chargeur d'amorçage
si l'on veut supporter le démarrage sur un {\em /boot} chiffré.

\paragraph{}
Le code est disponible sur {\em https://github.com/PFE-geom-luks}
