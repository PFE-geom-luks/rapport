\chapter{Etat de l'art}

\paragraph{}
Ce projet consiste en la réalisation d'un module noyau pour FreeBSD rendant
possible l'utilisation d'un volume chiffré avec
\textit{LUKS}\cite{onDiskFormatLuks}, un standard de chiffrement associé au
noyau \textit{Linux}.
\paragraph{}
L'objectif est donc de relever et comparer les moyens de chiffrement de disques
actuels, et d'analyser plus particulièrement les modules noyau Linux et FreeBSD
gérant le chiffrement de volume afin de porter le standard
\textit{LUKS}\cite{onDiskFormatLuks} vers FreeBSD.

\section{Différents niveaux de chiffrements}

\subsection{Chiffrement matériel}
\paragraph{}
Le chiffrement matériel (\textit{Hardware-based full disk encryption},
\textit{FDE}) concerne de nombreux disques durs (\textit{HDD}) et de plus en
plus de \textit{SSD}, alors appelés \textit{Self-encrypting drive} (ou
\textit{SED}). L'implémentation suit généralement la spécification \textbf{OPAL
  2.0} mise en place par le \textit{Trusted Computing Group}, réunissant
plusieurs fabricants d'espaces de stockage.
\paragraph{}
L'opération est directement réalisée par le contrôleur du disque qui œuvre
indépendamment du processeur. Il contient une clé de chiffrement servant à
chiffrer et déchiffrer le contenu du disque. Dès que le disque est alimenté,
cette clé est déchiffrée par ce contrôleur à l'aide d'une phrase secrète qui lui
est passée par un logiciel s'exécutant avant le démarrage du système ou via mot
de passe dans le BIOS. Cela donne la possibilité à l'utilisateur de modifier sa
phrase secrète sans risquer de perdre ses données ou devoir rechiffrer ses
données.

\subsection{Chiffrement logiciel de disque}
\paragraph{}
Le chiffrement de disque est généralement géré par le noyau du système, ou par
des logiciels spécialisés comme TrueCrypt ou VeraCrypt sur Windows. Il permet de
chiffrer et déchiffrer des partitions entières de manière transparente pour
l'utilisateur. Le disque, une fois
déchiffré, peut apparaître comme n'importe quel autre espace de stockage. Son
utilisation est totalement transparente pour l'utilisateur.
\paragraph{}
Dans des systèmes comme Linux ou FreeBSD, le chiffrement et déchiffrement du
disque est réalisé directement par le noyau. Au moment de la lecture et de
l'écriture du disque, les données passent par une couche supplémentaire
(\textit{dm-crypt} pour Linux et \textit{GELI}\cite{manGeli} pour FreeBSD) dont
le rôle est de faire le lien entre les données chiffrées du disque et celles en
clair : visibles et utilisables par l'utilisateur.
\paragraph{}
Le fait que la lecture d'un disque soit faite par le noyau permet à un système
utilisant cette méthode de chiffrer tout le système, à l'exception d'une petite
partie permettant de déchiffrer le noyau au démarrage. Le reste du système peut
ensuite être directement déchiffré par le noyau.

\subsection{Chiffrement au niveau du système de fichiers}
\paragraph{}
Le chiffrement de système de fichiers, ou système de fichiers chiffré, consiste
à confier le rôle de chiffrement et déchiffrement au sytème de fichiers, qui
peut alors chiffrer les fichiers, leurs métadonnées ainsi que l'arborescence,
selon les implémentations. Ce type de système de fichier chiffre alors
automatiquement les fichiers qui y sont stockés; le chiffrement se fait de manière
transparente pour l'utilisateur.
C'est notamment la technologie utilisée par l'\textit{Encrypting File
  System} (abrégé EFS), une technologie Microsoft disponible dès la troisième
version des systèmes de fichiers NTFS. D'autres solutions existent également,
comme EncFS. Des logiciels comme TrueCrypt/VeraCrypt permettent également de
réaliser ce genre d'opérations, en chiffrant des fichiers dans un conteneur.
\paragraph{}
Ce type de chiffrement a l'avantage de chiffrer toute une arborescence de
fichiers indépendamment des autres fichiers présents sur le disque. Cela permet
d'avoir différentes clés de chiffrement par système de fichiers chiffrés,
contrairement au chiffrement de disque qui en impose une seule par partition.
\paragraph{}
Ce type de chiffrement peut cohabiter sur un système utilisant déjà le
chiffrement de disque. Il ne permet en revanche pas de chiffrer des fichiers
nécessaires au démarrage du système, contrairement à l'autre type de
chiffrement.

\subsection{Chiffrement de fichier}
\paragraph{}
Ce type de chiffrement permet de chiffrer un seul fichier. Il peut s'agir
d'archives possédant cette propriété de pouvoir être chiffrées et déchiffrées
par des logiciels spécialisés. Dans ce cas, seul le contenu de l'archive est
chiffré et les métadonnées restent en clair. C'est le type de chiffrement
utilisé par des archives comme \textit{7z}. Le chiffrement peut également se
faire directement sur un fichier, en utilisant des algorithmes standards de 
chiffrement comme AES ou RSA, via des solutions logicielles comme
\textit{gpg}, le fichier ainsi créé par chiffrement ne reflète aucunement
le format du fichier en clair.


\section{Chiffrement de disque sur Linux et FreeBSD}

\paragraph{}
Le chiffrement de disque sous Linux et FreeBSD est un chiffrement qui s'effectue
de manière transparente pour l'utilisateur.
Il est géré par des modules noyaux qui interviennent au moment de la lecture et
l'écriture sur disque. Les opérations qu'ils réalisent sont totalement
transparentes du point de vue de l'utilisateur, qui ne manipule que des données
en clair.
\paragraph{}
Chacun des deux systèmes a accès à différents algorithmes implémentés dans le
noyau et permettant le chiffrement de données. Le chiffrement sur
les deux systèmes peut se faire sur des disques entiers, des partitions
physiques ou logiques (comme LVM), des volumes RAID, la RAM, ...
\paragraph{Linux}
Le module permettant de manipuler un disque chiffré sous Linux se nomme
\textit{dm-crypt}. Ce module a été ajouté à l'infrastructure
\textit{device-mapper} à la version 2.6 du noyau.\\
\\
Le standard qui est actuellement le plus utilisé pour le chiffrement de disque
est \textit{LUKS} (\textit{Linux Unified Key Setup}\cite{onDiskFormatLuks}). Son
rôle est de chiffrer intégralement un volume tout en le rendant utilisable par
d'autres plateformes.
\paragraph{FreeBSD}
Le module permettant cette opération sous FreeBSD se nomme \textit{GELI},
remplaçant l'ancien module \textit{GBDE}. Il appartient à la classe
\textit{GEOM}, qui contient l'ensemble des modules permettant de manipuler des
disques et autres volumes.\\
\\
Contrairement à \textit{dm-crypt} qui peut accepter des paramètres correspondant
à différents standards (non limités à LUKS), le seul standard disponible pour
le chiffrement de volume avec \textit{GELI} est celui de \textit{GELI} lui-même.

\subsection{Organisation dans le système d'exploitation}
\paragraph{}
Dans le cas de {\em Linux}, le chiffrement de disque est géré par le module 
noyau {\em dm-crypt}, qui est un module qui dépend du module plus général
{\em device-mapper} qui gère les transformations de disque sous {\em Linux} 
(RAID, LVM, Cache, ...).
Le module {\em dm-crypt} gère la table appelée {\em crypt table}
qui permet d'associer un device à un chiffrement notamment. La table permet
la lecture des données sur le disque. À cela s'ajoute le programme 
{\em cryptsetup} qui permet de lire des métadonnées stockées sur le disque 
pour remplir la {\em crypt table}.
Le programme s'occupe notamment du déchiffrement de la clé, de 
la lecture et de l'écriture des métadonnées sur le disque. L'organisation est 
donc la suivante :

\paragraph{}
\begin{figure}[h]
\centering
\includegraphics[width=.8\linewidth]{etat_art/organisation_linux.png}
\caption{\label{fig:OrgLinux}Organisation du code dans Linux}
\end{figure}

\paragraph{}
L'organisation reflétée par les programmes l'est également au niveau du code source,
le module {\em dm-crypt} n'a pas connaissance de l'existance de métadonnées ainsi 
que de leur format. On a donc le module {\em device-mapper \em et \em dm-crypt}
qui font parties du code du noyau {\em Linux}, tandis que {\em Crypsetup} est un 
programme qui est usuellement fourni avec {\em Linux} en tant qu'utilitaire, 
qui est géré par une équipe indépendante de celle du noyau {\em Linux}. De plus, 
{\em dm-crypt} n'a pas connaissance des algorithmes de chiffrement disponibles,
le module s'appuie sur l'API de cryptographie du noyau.

\paragraph{}
Dans le cas de {\em FreeBSD}, le module noyau qui est en charge des
transformations sur les disques est {\em GEOM}. Comme sous {\em Linux}, 
il existe des modules qui vont interagir avec {\em GEOM} pour fournir des 
fonctions de RAID, de cache, et également de chiffrement. 
Ces modules sont appelés {\em Classes GEOM}. 
Celle qui fournit le chiffrement est nommée {\em ELI}. Le module noyau associé 
est ainsi appelé {\em GELI}. Le module contrairement à {\em dm-crypt}, 
comprend toute la logique des métadonnées, ainsi que des algorithmes de 
chiffrement disponibles. L'outil en espace utilisateur tire son code source
directement des sources du module {\em geli}. {\em GELI} profite de la 
connaissance du format des métadonnées pour directement détecter les 
partitions chiffrées, et donc proposer leur déchiffrement à l'utilisateur, 
là où {\em Linux} ne peut pas détecter de
partition ou disque chiffré, et c'est à l'utilisateur de préciser qu'il veut 
déchiffrer tel disque avec l'outil {\em crypsetup} qui va donc reconnaître le 
format. On a donc l'organisation suivante sous FreeBSD:


\paragraph{}
\begin{figure}[h]
\centering
\includegraphics[width=.8\linewidth]{etat_art/organisation_FreeBSD.png}
\caption{\label{fig:OrgFreeBSD}Organisation du code dans FreeBSD}
\end{figure}

\subsection{Format sur disque}
\paragraph{}
Le format sur disque désigne l'organisation des données et métadonnées sur le 
disque. L'idée générale est de stocker sur une partie du disque les 
informations permettant de déchiffrer une autre partie du disque.
\paragraph{}
Comme le noyau {\em Linux} ne contient pas de format de métadonnées, 
c'est {\em Crypsetup} qui a choisi de créer un format appelé {\em Linux 
Unified Key Setup ({\em LUKS})}, comme format standard pour le chiffrement de 
disque sous {\em Linux}. En effet, {\em Crypsetup} implémente d'autres formats 
de métadonnées comme {\em TrueCrypt} développé par la 
{\em TrueCrypt Fondation}. 
Le format {\em LUKS} consiste en des métadonnées au début de la partition, 
précédant les données chiffrées.

\paragraph{}
\begin{figure}[h]
\centering
\includegraphics[width=.8\linewidth]{etat_art/format_disque_luks.png}
\caption{\label{fig:LUKSFormat}Format sur disque LUKS}
\end{figure}

\paragraph{}
Les métadonnées de {\em LUKS} contenant les algorithmes de chiffrement, 
d'authentification, la version de {\em LUKS}, sont suivis de huit emplacements 
permettant de stocker des clés chiffrées, qui permettent de déchiffrer le 
disque. Ainsi, on peut stocker la même clé huit fois, chacun protégées par des mots de 
passe différents, permettant ainsi à plusieurs utilisateurs de déchiffrer le 
même disque sans pour autant partager le même mot de passe.

\paragraph{}
Les métadonnées de {\em GELI} sont stockées contrairement à {\em LUKS}, 
à la fin du disque.
L'en-tête permet de stocker 2 clés, et s'étend sur un seul secteur.


\paragraph{}
\begin{figure}[h]
\centering
\includegraphics[width=.5\linewidth]{etat_art/format_disque_geli.png}
\caption{\label{fig:GELIFormat}Format sur disque GELI}
\end{figure}

\paragraph{}
Dans les deux en-têtes on retrouve des champs similaires, un {\em Magic} qui 
permet de connaître le type d'en-tête, la version, l'algorithme de chiffrement 
et d'authentification, etc. On peut noter tout de même que la taille de 
l'en-tête sous {\em GELI} fait 255 octets, contre 592 octets pour {\em LUKS}.



\subsection{Algorithmes}
\subsubsection{Algorithmes de chiffrement}
\paragraph{}
Les algorithmes de chiffrement disponibles pour {\em GELI} \cite{geli.h} sont :
\begin{itemize}
	\item Null
	\item Null-CBC
	\item AES-CBC
	\item AES-XTS
	\item Blowfish
	\item Blowfish-CBC
	\item Camellia
	\item Camellia-CBC
	\item 3DES
	\item 3DES-CBC
\end{itemize}

Cependant {\em GELI} profite des fonctions de chiffrement grâce à l'API noyau,
opencrypto, qui permettrait donc également d'utiliser les algorithmes suivants:
\begin{itemize}
	\item{CAST5-CBC}
	\item{SKIPJACK-CBC}
	\item{ARC4}
	\item{AES-ICM}
	\item{AES-GCM}
\end{itemize}

\paragraph{}
Dans le cas de {\em LUKS}, la spécification \cite{onDiskFormatLuks} précise
la liste des algorithmes disponibles, les noms des algorithmes ne sont pas 
interprétés par des opérations {\em LUKS}, mais pour des besoins de 
compatibilité avec d'autres implémentations, la liste des algorithmes 
disponibles est la suivante :
\begin{itemize}
	\item AES
	\item Twofish
	\item Serpent
	\item Cast5
	\item Cast6
\end{itemize}
Les algorithmes peuvent être associés aux modes suivants :
\begin{itemize}
	\item ECB
	\item CBC-plain
	\item CBC-essiv:hash
	\item XTS-plain64
\end{itemize}

\paragraph{}
On a donc la liste suivante qui peut être implémentée sur FreeBSD en utilisant 
l'API opencrypto :
\begin{itemize}
	\item{AES-CBC}
	\item{AES-XTS}
	\item{CAST5-CBC}
\end{itemize}


\subsubsection{Utilisation de la clé primaire}
\paragraph{}
Dans le cas de {\em LUKS}, la spécification \cite{onDiskFormatLuks} précise 
que la clé stockée dans l'en-tête {\em LUKS} est utilisée directement comme 
clé de chiffrement. Dans le cas de {\em GELI}, la clé est dérivée selon un 
algorithme déterministe en des clés utilisées pour chiffrer les données, 
qui dépendent de la version de {\em GELI} et de l'utilisation 
d'authentification des données \cite{manGeli}. 
La version 5 de {\em GELI} utilisait une seule clé, qui était la clé
primaire fournie par les métadonnées. A partir de la version 6, la clé primaire
est dérivée en clés de chiffrement, qui diffèrent tous les 2\textasciicircum20 
secteurs, soit 1048576 secteurs du disques. Les tailles des secteurs des 
disques étant généralement de 512 octets ou 4096 octets, 
donc tous les 500Mo ou 4Go environ. Ce changement de clé tous les
2\textasciicircum20 secteurs est en effet recommandé par le standard {\em IEEE 1619-2007}.
LUKS ne permet pas ce changement de clé tous les 2\textasciicircum20 secteurs,
puisque {\em dm-crypt} ne le permet pas.
La taille du secteur correspond au champ taille du secteur des métadonnées
{\em GELI}.

\subsubsection{Algorithmes d'authentification (HMAC)}
\paragraph{}
Les algorithmes d'authentification disponibles sous {\em GELI} sont:
\begin{itemize}
	\item HMAC/MD5
	\item HMAC/SHA1
	\item HMAC/Ripemd160
	\item HMAC/SHA256
	\item HMAC/SHA384
	\item HMAC/SHA512
\end{itemize}

\paragraph{}
Dans le cas de {\em LUKS}, le chiffrement avec de l'authentification des 
données n'est pas supporté. Cependant le service d'intégrité peut être rendu 
par le module {\em dm-verity} ou {\em dm-integrity}, le format des métadonnées 
est spécifié indépendamment de {\em LUKS}. Dans le cas où l'on souhaite faire 
du chiffrement et de l'authentification il est alors conseillé d'utiliser le 
module {\em dm-integrity}.

\subsubsection{Fonction pseudo aléatoire pour PBKDF2}
\paragraph{}
La spécification de PBKDF2 \cite{PKCS5v2} précise l'utilisation d'une fonction
pseudo aléatoire, dans le cas de {\em LUKS}, la spécification 
\cite{onDiskFormatLuks} liste les algorithmes autorisés
pour des raisons de compatibilité :
\begin{itemize}
	\item SHA1 (algorithme par défaut)
	\item SHA256
	\item SHA512
	\item Ripemd160
\end{itemize}
\paragraph{}
Dans le cas de {\em GELI}, c'est l'algorithme {\em SHA512} qui est utilisé
selon le code source \cite{geliPkcs5v2.c}.
\subsection{Fonctionnalités}
\paragraph{}
Les deux systèmes de chiffrement : {\em dm-crypt \em et \em geli} profitent 
de l'accélération matérielle pour les opérations de chiffrement grâce au 
noyau (AES-NI notamment). {\em LUKS} supporte l'hibernation sur disque 
contrairement à {\em GELI}, qui ne le supporte pas car {\em FreeBSD} ne 
supporte pas l'hibernation sur disque. Les deux systèmes permettent le 
renforcement de phrases de passe, via la fonction {\em PBKDF2} décrite 
par le standard {\em PKCS\#5v2}. Les deux formats de métadonnées 
permettent l'utilisation de plusieurs phrases de passe pour chiffrer la clé.

\subsection{Démarrage sur une partition chiffrée}
\paragraph{}
L'intégration avec le démarrage permet notamment d'avoir la racine du système 
de fichiers sur une partition chiffrée, et même éventuellement le {\em /boot}
dans lequel sont usuellement stockés les fichiers permettant de démarrer le 
système d'exploitation.

\paragraph{}
Le démarrage sur une partition chiffrée signifie que la partition contenant la
racine du système (contenant {\em/bin, /sbin, /etc, etc.}) est chiffrée et que
le démarrage sur ce système est possible.

\paragraph{}
On propose de rappeler la séquence de démarrage pour la suite.
\\
Le BIOS est le premier programme qui démarre et permet de lancer le chargeur d'amorçage, 
qui se charge de lancer le système, en lançant l'inital ramdisk. Ce dernier est un noyau
allégé, permettant de préparer notamment le disque lorsqu'il utilise LVM, du 
RAID, ou encore du chiffrement. Le noyau contient le minimum et les modules
nécessaires sont chargés au démarrage. Cela permet de réduire la taille du noyau
à charger en RAM, tout en supportant un grand nombre de matériels. L'initial
ramdisk se charge donc de charger les modules qui correspondent au matériel de
la machine, puis lance le premier processus appelé Init qui se charge de lancer
les autres processus à lancer au démarrage.


\begin{figure}[h]
\centering
\includegraphics[width=.2\linewidth]{etat_art/Demarrage.png}
\caption{\label{fig:demarrage}Démarrage sous Linux et FreeBSD}
\end{figure}

\paragraph{}
Comme il existe plusieurs cas possibles de partitionnement, on détaille dans la
suite les principales solutions existantes.
\\
Sous Linux comme sous FreeBSD, la partition contenant la racine du système ne
contient pas nécessairement le dossier {\em /boot} qui peut être stocké sur une
partition différente. Il y a alors deux cas, soit la partition {\em /boot} est
chiffrée, soit elle ne l'est pas. Le premier cas est alors identique à une seule
partition chiffrée contenant également le {\em /boot} (en ignorant les
différences concernant le système de fichiers ainsi que la possiblité de séparer
d'autres dossiers sur des partitions différentes). Dans le cas d'un {\em /boot}
en clair, il est plus courant de trouver une configuration avec deux partitions,
une contenant le {\em /} et une autre le {\em /boot}.

\subsubsection{Démarrage sur une partition chiffrée sous Linux avec LUKS}

\paragraph{}
Sous Linux, dans le cas d'un {\em /boot} séparé, le système démarre normalement,
puis lorsque l'Initrd est chargé, il charge les modules et lance les scripts
(nommés hooks) nécessaires pour démarrer, dont ceux permettant de monter la
racine. Un script est dédié au déchiffrement des disques, son nom et son chemin
dans le système de fichiers varie selon les distributions. Le script est donc
lancé et permet de déchiffrer la partition chiffrée et donc le démarrage
peut continuer normalement.

\paragraph{}
Dans le cas où le {\em /boot} est chiffré, c'est le rôle des premières parties
du chargeur d'amorçage de déchiffrer le disque pour pouvoir charger l'Initrd. Parmi les
chargeurs d'amorçage supportés par Linux on peut citer GRUB2 et LILO
(qui n'est plus maintenu) qui supportent le déchiffrement de partition chiffrée
avec LUKS.

\begin{figure}[h]
\centering
\includegraphics[width=.9\linewidth]{etat_art/DemarrageLinux.png}
\caption{\label{fig:demarrageLinux}Démarrage sur une partition chiffrée avec LUKS}
\end{figure}

\paragraph{}

Il faut bien noter la différence entre les deux cas. Dans le premier, un script
s'occupe de déchiffrer le disque en utilisant le module noyau dm-crypt qui sera
chargé, et donc le script sera en mesure de déchiffrer n'importe quel type de
volume chiffré compatible avec dm-crypt. Dans le second cas, le
déchiffrement du disque n'est possible que pour des disque chiffrés avec LUKS.
Pour l'instant seul le format 1.0 de LUKS est d'ailleurs supporté par GRUB2.

\subsubsection{Démarrage sur une partition chiffrée sous FreeBSD avec GELI}

\paragraph{}
Dans le cas de FreeBSD avec un {\em /boot} séparé, c'est tout simplement au
noyau de déchiffrer le disque pour le monter. Cela est réalisé en chargant
le module {\em geom\_eli}, qui va alors chercher toutes les partitions qui sont
de type {\em GELI} et va proposer de les déchiffrer. Cette étape est
équivalente à celle de l'Initrd de Linux.

\paragraph{}
Dans le cas d'une seule partition contenant le {\em /} et le {\em /boot}, comme
la chaîne de démarrage comprend plusieurs étapes, c'est d'abord à {\em boot2}
(la dernière étape du chargeur d'amorçage) de déchiffrer le disque pour charger le chargeur
de FreeBSD. Ce dernier va alors avoir besoin de déchiffrer le disque pour pouvoir
utiliser les noyaux disponibles notamment. Enfin dans le noyau, le module
{\em geom\_eli} va permettre de déchiffrer la partition contenant la racine.
Pour éviter d'avoir à taper la phrase de passe plusieurs fois, celle-ci
est passée via des zones mémoires partagées aux différents parties :
{\em boot2} donne la phrase de passe au {\em chargeur} qui la donne au noyau.


\paragraph{}
\begin{figure}[h]
\centering
\includegraphics[width=.9\linewidth]{etat_art/DemarrageFreeBSD.png}
\caption{\label{fig:demarrageFreeBSD}Démarrage sur une partition chiffrée avec GELI}
\end{figure}

\paragraph{}
On constate donc que {\em LUKS} standardise le format sur disque et restreint
les algorithmes possibles. Les différences avec {\em GELI} se trouvent donc
sur le format des métadonnées sur le disque, les algorithmes disponibles.
{\em LUKS} ne prévoit pas l'authentification des données
contrairement à {\em GELI}.

On peut donc partir du code source de GELI et essayer d'adapter pour pouvoir
utiliser le format {\em LUKS}.
