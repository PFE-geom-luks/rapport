\chapter{Description du projet}

L'analyse de \textit{LUKS} et \textit{GELI} nous a montré les similutudes et
différences que présentent ces deux systèmes.

\section{Lecture des métadonnées}
\paragraph{}
Pour n'importe-quel module noyau utilisant la classe \textit{GEOM} de
\textit{FreeBSD}, les métadonnées sont toujours placées à la fin des données du
disque, contrairement à \textit{LUKS}. Il faut donc créer les fonctions
utilisant celles fournies par \textit{GEOM} afin de réaliser cette opération.
\paragraph{}
Dans le cas d'un fichier, il suffit d'utiliser les fonctions standards de
\verb|C| et de lire le début du fichier. Dans le cas d'un disque en revanche, il
faut créer une fonction lisant les premiers secteurs. On peut ensuite les
stocker dans une structure afin de connaître les propriétés de chiffrement
utilisées.

\section{Utilisation des métadonnées}
\paragraph{}
Les informations présentes dans les métadonnées \textit{LUKS} sont assez proches
de celles de \textit{GELI}:
\begin{itemize}
\item le \textit{MAGIC}
\item la version utilisée
\item l'algorithme de chiffrement
\item l'algorithme de hashage
\item le sel
\item le nombre d'itérations PKCS
\item la \textit{masterkey} chiffrée
\end{itemize}
D'autres paramètres comme la longueur de clé ou le type de vecteur
d'initialisation \textit{IV} sont dépendant des spécification \textit{LUKS} ou
\textit{GELI}.
\paragraph{}
Écrire un module noyau de toute pièce capable d'utiliser un disque chiffré avec
\textit{LUKS} serait une tâche très longue. Sachant que de nombreuses fonction
s'inspireraient du module \textit{GELI}, il est préférable de copier le code
(sous license \underline{BSD-2-Clause-FreeBSD}) et de l'adapter à la lecture et
l'écriture de données sur un disque chiffré \textit{LUKS}. Cependant, les
fonctions étant extrêmenent dépendantes de la structure des métadonnées, leur
modification équivaudrait à une refonte entière du code.
\paragraph{}
C'est pourquoi, la solution la plus intéressante serait de garder la structure
de \textit{GELI} et d'y mettre les informations de chiffrement
Il faudra toutefois faire attention à l'emplacement des métadonnées sur le
disque qui change entre les deux spécifications.
