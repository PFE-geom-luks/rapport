\section{Analyse de GELI}

\paragraph
On constate donc que {\em LUKS} standardise le format sur disque et restreint 
les algorithmes possibles. Les différences avec {\em GELI} se trouvent donc 
sur le format des métadonnées sur le disque, les algorithmes disponibles.
{\em LUKS} ne prévoit pas l'authentification des données 
contrairement à {\em GELI}.

On peut donc partir du code source de GELI et essayer d'adapter pour pouvoir 
utiliser le format {\em LUKS}.

\subsection{Analyse des fichiers sources de \em GELI}
\subsubsection{Module noyau : sys/geom/eli/}
\paragraph{g\_eli.c}
\begin{itemize}
	\item fetch\_loader\_passphrase : récupère la phrase de passe depuis 
		l'environnement noyau qui a été insérée par le loader 
		(utile pour démarrer sur une partition chiffrée)
	\item zero\_boot\_passcache : rempli de zero la phrase de passe cachée par le 
		système
	\item zero\_geli\_intake\_keys : rempli de zero le buffer de clés
	\item zero\_intake\_passcache : appelle les deux fonctions précédentes
	\item g\_eli\_crypto\_rerun : relance les opérations de chiffrement (EAGAIN from
		crypto(9) )
	\item g\_eli\_getattr\_done : 
	\item g\_eli\_read\_done : appellée lorsque les données chiffrée sont lues sur le 
		fournisseur (disque), elle appelle si tout c'est bien passé la partie
		qui va déchiffrer (en ajoutant à une queue les données à déchiffrer)
	\item g\_eli\_write\_done : appellée lorsque les données ont été écrites sur le 
		disque
	\item g\_eli\_orphan\_spoil\_assert : déclenche une panique noyau (ne devrait pas
		être appellée)
	\item g\_eli\_orphan : appellée quand l'un des parents de l'instance g\_eli 
		devient absent, appelle g\_eli\_destroy si les métadonnées sont présentes
	\item g\_eli\_start : appellée lorsque on commence une lecture ou écriture, 
		filtre les IO (READ,WRITE,GETATTR,FLUSH, etc)
	\item g\_eli\_newsession : initialise les fonctions de chiffrements (workers)
	\item g\_eli\_freesession : libère les fonctions de chiffrement noyau (workers)
	\item g\_eli\_cancel : renvoie ENXIO : pas de tel adresse ou matériel: empèche
		la création de nouvelles IO
	\item g\_eli\_takefirst : prise des IO dans la queue (vérifie que le disque n'est
		pas suspendu)
	\item g\_eli\_worker : fonction principale du thread worker noyau quand il n'y a
		pas d'accélération matérielle.
	\item g\_eli\_read\_metadata : lis les métadonnées sur le disque
	\item g\_eli\_last\_close :
	\item g\_eli\_access : gère les droits d'accès (lecture, écriture, exclusivité)
	\item g\_eli\_cpu\_is\_disabled
	\item g\_eli\_create : initialise l'instance geli (lecture des métadonnées, etc)
		créé le disque déchiffré
	\item g\_eli\_destroy : supprime le disque déchiffré
	\item g\_eli\_destroy\_geom : appelle la fonction précédente
	\item g\_eli\_keyfiles\_load : charge les keyfiles
	\item g\_eli\_keyfiles\_clear : supprime les keyfiles de la mémoire vive
	\item g\_eli\_taste : fonction de détection d'un partition GELI que l'on réussi à
		ouvrir : la phrase de passe et/ou keyfiles sont corrects
	\item g\_eli\_dumpconf : permet à l'utilisateur de connaître des données sur 
		l'instance actuelle : version, algorithmes, etc
	\item g\_eli\_shutdown\_pre\_sync : 
	\item g\_eli\_init : enregistre l'événement shutdown qui déclenche la fonction
		précédente
	\item g\_eli\_fini : enlève l'événement shutdown
\end{itemize}

\paragraph{g\_eli.h}
\begin{itemize}
	\item eli\_metadata\_encode\_v*
	\item eli\_metadata\_decode\_v*
	\item eli\_metadata\_decode,eli\_metadata\_encode : choisi la bonne fonction pour 
		respectivement le décodage et l'encodage des métadonnées
	\item g\_eli\_str2ealgo : convertie une chaine de caractère en algorithme de 
		chiffrement
	\item g\_eli\_str2aalgo : convertie une chaine de caractère en algorithme 
		d'authentification
	\item g\_eli\_keylen : retourne la longueur de la clé selon l'algorithme choisi
	\item g\_eli\_hashlen : retourne la longeur du hash selon l'algorithme utilisé
	\item eli\_metadata\_softc : rempli les données privées de l'instance geli 
		selon les métadonnées extraites du disque
\end{itemize}

\paragraph{g\_eli\_crypto.c}
\begin{itemize}
	\item g\_eli\_crypto\_done : 
	\item g\_eli\_crypto\_cipher : création du worker qui s'occupe du 
		chiffrement
	\item g\_eli\_crypto\_encrypt, g\_eli\_crypto\_decrypt : 
		renvoie la fonction de chiffrement pour les métadonnées
\end{itemize}

\paragraph{g\_eli\_ctl.c}
\begin{itemize}
	\item g\_eli\_ctl\_attach : ouvrir un volume {\em GELI}
	\item g\_eli\_find\_device : renvoie les données du module noyau 
		associé au disque /dev/sdAN
	\item g\_eli\_ctl\_detach : ferme un volume {\em GELI}
	\item g\_eli\_ctl\_onetime : créée un volume {\em GELI} avec une clé
		aléatoire
	\item g\_eli\_ctl\_configure : configure les différentes options des
		volumes chiffrés avec {\em GELI}
	\item g\_eli\_ctl\_setkey : change la clé de chiffrement des 
		métadonnées.
	\item g\_eli\_ctl\_delkey : supprime une clé de chiffrement des 
		métadonnées
	\item g\_eli\_suspend\_one : suspend le volume chiffré
	\item g\_eli\_ctl\_suspend : suspend le volume chiffré en appelant la 
		fonction précédente
	\item g\_eli\_ctl\_resume : reprend un volume chiffré
	\item g\_eli\_kill\_one : supprime les métadonnées sur le disque
	\item g\_eli\_ctl\_kill : supprime les métadonnées sur le disque en 
		appelant la fonction précédente
	\item g\_eli\_config : interprète la commande passée
\end{itemize}
\paragraph{g\_eli\_hmac.c}
\begin{itemize}
	\item g\_eli\_crypto\_hmac\_init : initialisation du HMAC
	\item g\_eli\_crypto\_hmac\_update : passage des données à l'algorithme
		de HMAC
	\item g\_eli\_crypto\_hmac\_final : réalise les dernière opérations 
		pour calculer le HMAC
	\item g\_eli\_crypto\_hmac : réalise le HMAC des données en appelant 
		les 3 fonctions précédentes
	\item g\_eli\_crypto\_ivgen : génère un vecteur d'initialisation unique
		pour chaque secteur
\end{itemize}

\paragraph{g\_eli\_integrity.c}
Le code source de ce fichier est uniquement utile pour les volumes chiffrés qui
authentifient également les données pour pouvoir vérifier leur intégrité.
Il ne nous est donc pas utile pour porter LUKS.
\paragraph{g\_eli\_key.c}
\begin{itemize}
	\item g\_eli\_mkey\_verify : vérifie que la clé maitres est la bonne en
		la comparant au HMAC trouvé sur le disque.
	\item g\_eli\_mkey\_hmac : calcule le HMAC de la clé des données et des
		vecteurs d'initialisation.
	\item g\_eli\_mkey\_decrypt : déchiffre la clé maitre
	\item g\_eli\_mkey\_encrypt : chiffre la clé maitre
	\item g\_eli\_mkey\_propagate : initialise les clé de chiffrements
\end{itemize}
\paragraph{g\_eli\_key\_cache.c}
\begin{itemize}
	\item g\_eli\_key\_cmp : compare deux clés
	\item g\_eli\_key\_fill : 
	\item g\_eli\_key\_allocate : alloue une clé de chiffrement et la 
		stocke dans une arbre et une queue
	\item g\_eli\_key\_find\_last : renvoie la dernière clé ajoutée
	\item g\_eli\_key\_replace : remplace une clé dans la queue et l'arbre
	\item g\_eli\_key\_remove : supprime la clé de la queue et de l'arbre
	\item g\_eli\_key\_init : initialise l'arbre et la queue
	\item g\_eli\_key\_destroy : spprime l'arbre et la queue
	\item g\_eli\_key\_hold : met une clé de chiffrement dans le cache
	\item g\_eli\_key\_drop : appelle la fonction g\_eli\_key\_remove
\end{itemize}

\paragraph{g\_eli\_privacy.c}
\begin{itemize}
	\item g\_eli\_crypto\_read\_done : récupère la donnée déchiffrée, et la
		renvoie
	\item g\_eli\_crypto\_write\_done : renvoie les données chiffrées 
		au disque pour écriture
	\item g\_eli\_crypto\_read : lance la lecture sur le disque
	\item g\_eli\_crypto\_run : chiffre ou déchiffre des données
\end{itemize}

\paragraph{pkcs5v2.c}
\begin{itemize}
	\item
	\item
	\item
\end{itemize}
